%% ============================================================================
%%
%% PLATO MSAP5 documentation
%%
%% Templatenauthor: Jakob Lysgaard Rørsted [jakob (at) phys (dot) au (dot) dk]
%%
%% Module MSAP3-31
%% ============================================================================
% ~~~~~~~~~~~~~~~~~~~~~~~~~~~~~~~~~~~~~~~~~~~~~~~~~~~~~~~~~~~~~~~~~~~~~~~~~~~~~
% Preamble
% ~~~~~~~~~~~~~~~~~~~~~~~~~~~~~~~~~~~~~~~~~~~~~~~~~~~~~~~~~~~~~~~~~~~~~~~~~~~~~
\documentclass[a4paper, oneside, 11pt, article, english]{memoir}

\usepackage{natbib}
\bibliographystyle{unsrtnat}


%\usepackage{hyperref}


% Input standard preamble
%\input{preamble}
\usepackage{xcolor}
\usepackage[colorlinks]{hyperref}

\def\tmp#1#2#3{%
  \definecolor{Hy#1color}{#2}{#3}%
  \hypersetup{#1color=Hy#1color}}
\tmp{link}{HTML}{800006}
\tmp{cite}{HTML}{2E7E2A}
\tmp{file}{HTML}{131877}
\tmp{url} {HTML}{8A0087}
\tmp{menu}{HTML}{727500}
\tmp{run} {HTML}{137776}
\def\tmp#1#2{%
  \colorlet{Hy#1bordercolor}{Hy#1color#2}%
  \hypersetup{#1bordercolor=Hy#1bordercolor}}
\tmp{link}{!60!white}
\tmp{cite}{!60!white}
\tmp{file}{!60!white}
\tmp{url} {!60!white}
\tmp{menu}{!60!white}
\tmp{run} {!60!white}

% Information for maketitle
\author{Earl Patrick Bellinger}
\newcommand\shorttitle{Documentation of MSAP3-34\xspace}  % For the header
\title{\shorttitle{} for PLATO \\ \Large\textit{Validation}}
\date{\today}


% ~~~~~~~~~~~~~~~~~~~~~~~~~~~~~~~~~~~~~~~~~~~~~~~~~~~~~~~~~~~~~~~~~~~~~~~~~~~~~
% Content
% ~~~~~~~~~~~~~~~~~~~~~~~~~~~~~~~~~~~~~~~~~~~~~~~~~~~~~~~~~~~~~~~~~~~~~~~~~~~~~
\begin{document}
\maketitle

% Introductionary tables I: Author(s)
\begin{table}[htbp]
  \centering
  \caption{Author information}
  \label{tab:author}
  \begin{tabular}{cc}
    \toprule
    Prepared by & Date\\
    \midrule
    Earl P. Bellinger & March 4, 2023 \\
    \\
    Checked by \\
    \midrule
    \\
    Approved by \\
    \midrule
    \\
    Authorized by \\
    \midrule
    \\
    \bottomrule
  \end{tabular}
\end{table}

% Introductionary tables II: Version
\begin{table}[htbp]
  \centering
  \caption{Version history}
  \label{tab:version}
  \begin{tabular}{ccccc}
    \toprule
    Issue & Date & \textnumero{} change description & Page(s) & Paragraph(s) \\
    \midrule
    1.0 & March 4, 2023 & Initial release & All & All \\
    1.1 & January 12, 2024 & Updated version & All & All \\
    1.2 & March 4, 2024 & Added full seismic inputs & All & All \\
    \bottomrule
  \end{tabular}
\end{table}


% ~~~~~~~~~~~~~~~~~~~~~~~~~~~~~~~~~~~~~
% Table of contents
% ~~~~~~~~~~~~~~~~~~~~~~~~~~~~~~~~~~~~~
\clearpage
\tableofcontents*
\clearpage


% ~~~~~~~~~~~~~~~~~~~~~~~~~~~~~~~~~~~~~
% Introduction
% ~~~~~~~~~~~~~~~~~~~~~~~~~~~~~~~~~~~~~
\chapter{Introduction}
\label{chap:intro}

\section{Scope of the document}
\label{sec:scope}

This document aims to provide a description of the validation algorithm for the selection and validation module of the MSAP5. 
It provides technical details (inputs, outputs, data types) as well as the
functional description (implementation). 
%The justification for the choice of this specific algorithm and a description of its scientific performances is provided in [please provide the reference of the justification document]. 
Moreover, the exact position of this algorithm within the data processing pipeline is described in [RD1]. 


\section{Nomenclature}
\label{sec:nomenclature}

% TEXT FROM TEMPLATE BELOW IN emph OR itshape
%\emph{Please provide in the table, the definition of the technical terms when
%undefined in the text.}

See \ref{tab:nomenclature} and \ref{tab:datatypes}.

% Table with nomenclature
\begin{table}[htbp]
  \centering
  \caption{Nomenclature}
  \label{tab:nomenclature}
  \begin{tabular}{@{}lp{9cm}@{}}
    \toprule
    Term                & Description                                                                                                                                                                                                                                                                                                                                                     \\
    \midrule
    M & mass of the star in units of the solar mass M$_\odot$ \\
%  \item[M$_\odot$] solar mass
    R & radius of the star in units of the solar radius R$_\odot$ \\
%  \item[M$_\odot$] solar radius
    A & age of the star in units of Gyr \\
    %Baseline algorithms & initial, minimum set of algorithms needed to perform the essential scientific analysis tasks of the EAS or SAS pipelines. As they are the 'initial' (or 'baseline') set of algorithms, they could be updated later on. By 'essential' scientific analysis we mean the analysis / data processing that is necessary for fulfilling the scientific goals of PLATO.\\
    %Legacy code         & it is defined as a code that has been previously developed, used, and validated in contexts other than PLATO. To be used in the PLATO framework, it must be well-documented, extensively tested, and too heavy to be implemented from scratch by the PDC.                                                                                                       \\
    %Prototype           & preliminary software that can help testing the algorithm, specially input and output data                                                                                                                                                                                                                                                                       \\
    %Pseudo-code         & a schematic description of the algorithm or code written in a simplified language                                                                                                                                                                                                                                                                               \\
    %{[}tbc{]}           & {[}tbc{]}                                                                                                                                                                                                                                                                                                                                                       \\
    \bottomrule
  \end{tabular}
\end{table}

% Table with data types
\begin{table}[htbp]
  \centering
  \caption{Standard data types}
  \label{tab:datatypes}
  \begin{threeparttable}
    \begin{tabular}{@{}lll@{}}
      \toprule
      Type             & Size               & Values                                          \\
      \midrule
      array & arbitrary & floats \\
      %
      %bool             & 1 bit              & 0 (false) 1 (true)                              \\
      %enumeration      &                    & Label 1 = value 1, Label 2 = value 2, ...       \\
      %signed byte      & 8 bits             & {[}-128, 127{]}                                 \\
      %signed short     & 16 bits            & {[}–32768, 32767{]}                             \\
      %signed int       & 32 bits            & {[}-2147483648, 2147483647{]}                   \\
      %signed long      & 64 bits            & {[}–9223372036854775808, 9223372036854775807{]} \\
      %unsigned byte    & 8 bits             & {[}0,255{]}                                     \\
      %unsigned short   & 16 bits            & {[}0,65535{]}                                   \\
      %unsigned int     & 32 bits            & {[}0,4294967295{]}                              \\
      %unsigned long    & 64 bits            & {[}0,18446744073709551615{]}                    \\
      %float\tnote{a}   & 32 bits            & 3.4E +/- 38 (7 digits)                          \\
      %double\tnote{b}  & 64 bits            & 1.7E +/-308 (15 digits)                         \\
      %string           & 128 char\tnote{c}  &                                                 \\
      \bottomrule
    \end{tabular}
    \begin{tablenotes}
    %\item[a:] float with single precision
    %\item[b:] float with double precision
    %\item[c:] 1 char = 1 byte
    \end{tablenotes}
  \end{threeparttable}
\end{table}


\section{Referenced documents}
\label{sec:docs}

The following documents are referenced:

\begin{description}
  \firmlist
\item[RD1] PLATO-LESIA-PSPM-DD-0021, Work and data flows of the stellar L1/L2 processing pipeline
\item[MSAP3-31] PLATO-MSAP3-31, Consistency checks
\item[MSAP3-32] PLATO-MSAP3-32, Selection
\end{description}



\section{Abbreviations}
\label{sec:abbrev}

% TEXT FROM TEMPLATE BELOW IN emph OR itshape
%\emph{To be adapted and completed}

\begin{description}
  \firmlist
  %\item[HSD] Tukey's Honestly Significantly Different statistical test
%  \item[IDP] Intermediate Data Product
%  \item[Gyr] one billion years
%\item[ATBD] Algorithm Theoretical Baseline Document
%\item[DO-SAPP] During Operation Stellar Abundances and atmospheric Parameters Pipeline DP Data Product
%\item[EAS] Exoplanet Analysis System
%\item[ESA] European Space Agency
%\item[FU] follow-up
%\item[GO] Guest Observer
%\item[GOP] Ground-based Observation Program
%\item[LC] light curve
%\item[PLATO] PLAnetary Transits and Oscillations of Stars
%\item[PDC] PLATO Data Centre
%\item[PDC-DB] PDC-Database
%\item[PIC] PLATO Input Catalogue
%\item[PMC] PLATO Mission Consortium ppm parts per million
%\item[PSM] PLATO Science Management
%\item[SAS] Stellar Analysis System
%\item[SNR] Signal-to-Noise Ratio
%\item[TBC] To Be Confirmed
%\item[TBD] To Be Defined
%\item[TBS] To Be Specified
%\item[TBW] To Be Written
%\item[URD] User Requirements Document
%\item[URJD] User Requirements Justification Document
\end{description}


% ~~~~~~~~~~~~~~~~~~~~~~~~~~~~~~~~~~~~~
% Overview
% ~~~~~~~~~~~~~~~~~~~~~~~~~~~~~~~~~~~~~
\clearpage
\chapter{General overview}
\label{chap:overview}

\section{Name of the algorithm and status}
\label{sec:name}
The algorithm is MSAP5-34, \emph{Validation}. 
The baseline algorithm has been implemented, but revisions are expected. 
In particular, this algorithm currently works with the publically available MIST stellar tracks \citep{MIST-I, MIST-II}, but needs to be updated to use the official stellar models used in the PLATO mission. 

% TEXT FROM TEMPLATE BELOW IN emph OR itshape
%\emph{The name of the algorithm should follow the convention defined in the SAS
%architecture document {RD1}}

%\emph{Please specify if the algorithm is part of the baseline and if it is expected
%to be a first version intended to be updated or likely to be the final one.}


\section{Synopsis}
\label{sec:synopsis}

The objective of MSAP5-34 is to validate the selected values of mass (M), radius (R), and age (A). 
In particular, we check whether these values are physically consistent with stellar models. 
We achieve this with a $\chi^2$ statistical test in which we reject the null hypothesis at a significance level of $0.01$. 
If the validation check succeeds (i.e., we fail to reject the null hypothesis), then we return the mean, standard deviation, and source of each measurement. 


% TEXT FROM TEMPLATE BELOW IN emph OR itshape
%\emph{Please provide a short description of the algorithm objective(s)}

%\emph{Please provide the reference to the technical note that justifies the selection
%of the algorithm. The later shall be sent in a separate document to WP12.}


\section{Model}
\label{sec:model}

We find the model in the grid with the smallest $\chi^2$, defined as:
\begin{equation}
    \chi^2 
    = 
    \frac{(M - \hat{M})^2}{\sigma_M^2}
    +
    \frac{(R - \hat{R})^2}{\sigma_R^2}
    +
    \frac{(A - \hat{A})^2}{\sigma_A^2}.
\end{equation}
Here $M, R, A$ are the mean values of the selected measurements provided by MSAP5-32; $\sigma_M, \sigma_R, \sigma_A$ are their standard deviations; and $\hat{M}, \hat{R}, \hat{A}$ are the theoretical values from the grid of models. 
We then use \texttt{scipy 1.10.1} to compute the $\chi^2$ test statistic. 

The module outputs None for each measurement if they are either inconsistent (from the previous module) or invalid (as determined by this module). 



% TEXT FROM TEMPLATE BELOW IN emph OR itshape
%\emph{Please provide here a short description of the main steps of the algorithm}


% ~~~~~~~~~~~~~~~~~~~~~~~~~~~~~~~~~~~~~
% Input / output
% ~~~~~~~~~~~~~~~~~~~~~~~~~~~~~~~~~~~~~
\clearpage
\chapter{Lists of inputs and outputs}
\label{chap:inputoutput}

\section{Complete list of inputs}
\label{sec:input}

See documentation for MSAP5-31. 

\iffalse
The inputs are tabulated in Table~\ref{tab:input}. 

\iffalse
% TEXT FROM TEMPLATE BELOW IN emph OR itshape
{
  \itshape
  %A comprehensive list of inputs must be provided in the table. For each input,
  %the following information is needed

  \begin{description}
    \firmlist
  \item[Name] the name must follow the nomenclature as defined by WP120 Data
    Products Definition Document [RD3].
  \item[Source] module or sub-module from which the data is generated (e.g.,
    database, or previous module/sub-module output parameter). Also specify if the
    data originates from the current quarter (default) or from a previous quarter.
  \item[Status] specify if this data is \emph{mandatory} or \emph{optional} to run the algorithm.
  \item[Data type] see \cref{sec:nomenclature} for the standard definitions
  \item[Dimension] specify the dimension of the data (e.g; the dimension of a scalar is 0, of an array 1, etc).
  \item[Unit] provide the unit of the data and the data-system of units (cgs or mks).
  \end{description}
}
\fi

\begin{table}[htbp]
  \centering
  \caption{Input parameters}
  \label{tab:input}
  \begin{tabular}{lccccc}
    \toprule
    Name & Source & Status & Data type & Dimension & Unit \\
    \midrule
    \midrule
    IDP\_SAS\_AGE\_GYRO                  & IDP\_125 & optional & array & arbitrary & Gyr \\
    IDP\_SAS\_AGE\_ACTIVITY              & IDP\_125 & optional & array & arbitrary & Gyr \\
    IDP\_SAS\_MASS\_GRANULATION          & IDP\_125 & optional & array & arbitrary & M$_\odot$ \\
    IDP\_SAS\_MASS\_GRANULATION\_CGBM    & IDP\_125 & optional & array & arbitrary & M$_\odot$ \\
    IDP\_SAS\_RADIUS\_GRANULATION\_CGBM  & IDP\_125 & optional & array & arbitrary & R$_\odot$ \\
    IDP\_SAS\_AGE\_GRANULATION\_CGBM     & IDP\_125 & optional & array & arbitrary & Gyr \\
    %IDP\_SAS\_MASS\_RHO\_TRANSIT\_CGBM   & IDP\_125 & optional & array & arbitrary & M$_\odot$ \\ 
    %IDP\_SAS\_RADIUS\_RHO\_TRANSIT\_CGBM & IDP\_124 & optional & array & arbitrary & R$_\odot$ \\
    %IDP\_SAS\_AGE\_RHO\_TRANSIT\_CGBM    & IDP\_124 & optional & array & arbitrary & Gyr \\ 
    \midrule 
    IDP\_SAS\_MASS\_SCALING\_ONLY        & IDP\_124 & optional & array & arbitrary & M$_\odot$ \\
    IDP\_SAS\_RADIUS\_SCALING\_ONLY      & IDP\_124 & optional & array & arbitrary & R$_\odot$ \\
    IDP\_SAS\_MASS\_SCALING\_GRIDS       & IDP\_124 & optional & array & arbitrary & M$_\odot$ \\
    IDP\_SAS\_RADIUS\_SCALING\_GRIDS     & IDP\_124 & optional & array & arbitrary & R$_\odot$ \\
    IDP\_SAS\_AGE\_SCALING\_GRIDS        & IDP\_124 & optional & array & arbitrary & Gyr \\
    IDP\_SAS\_MASS\_GRID\_MIXED          & IDP\_124 & optional & array & arbitrary & M$_\odot$ \\
    IDP\_SAS\_RADIUS\_GRID\_MIXED        & IDP\_124 & optional & array & arbitrary & R$_\odot$ \\
    IDP\_SAS\_AGE\_GRID\_MIXED           & IDP\_124 & optional & array & arbitrary & Gyr \\
    IDP\_SAS\_MASS\_GRID\_SURF\_IND      & IDP\_124 & optional & array & arbitrary & M$_\odot$ \\
    IDP\_SAS\_RADIUS\_GRID\_SURF\_IND    & IDP\_124 & optional & array & arbitrary & R$_\odot$ \\
    IDP\_SAS\_AGE\_GRID\_SURF\_IND       & IDP\_124 & optional & array & arbitrary & Gyr \\
    IDP\_SAS\_MASS\_FREQS                & IDP\_124 & optional & array & arbitrary & M$_\odot$ \\
    IDP\_SAS\_RADIUS\_FREQS              & IDP\_124 & optional & array & arbitrary & R$_\odot$ \\
    IDP\_SAS\_AGE\_FREQS                 & IDP\_124 & optional & array & arbitrary & Gyr \\
    \midrule 
    IDP\_SAS\_METADATA\_SCALING\_ONLY        & IDP\_124 & optional & string & arbitrary &  \\
    IDP\_SAS\_METADATA\_SCALING\_GRIDS       & IDP\_124 & optional & string & arbitrary &  \\
    IDP\_SAS\_METADATA\_GRID\_MIXED          & IDP\_124 & optional & string & arbitrary &  \\
    IDP\_SAS\_METADATA\_GRID\_SURF\_IND      & IDP\_124 & optional & string & arbitrary &  \\
    IDP\_SAS\_METADATA\_GRID\_FREQS          & IDP\_124 & optional & string & arbitrary &  \\
    \midrule
    IDP\_SAS\_MASS\_PRIORITY & mandatory & string & 1 & N/A \\
    IDP\_SAS\_RADIUS\_PRIORITY & mandatory & string & 1 & N/A \\
    IDP\_SAS\_AGE\_PRIORITY & mandatory & string & 1 & N/A \\
    %\fi
    \bottomrule
  \end{tabular}
\end{table}
\fi

\section{Complete list of outputs}
\label{sec:output}

\iffalse
% TEXT FROM TEMPLATE BELOW IN emph OR itshape
{
  \itshape
  A comprehensive list of outputs must be provided in the table. For each output,
  the following information is needed

  \begin{description}
    \firmlist
  \item[Name] the name must follow the nomenclature as defined by WP120 Data
    Products Definition Document [RD3].
  \item[Status] specify if this data is \emph{mandatory} or \emph{optional} to run the algorithm.
  \item[Data type] see \cref{sec:nomenclature} for the standard definitions
  \item[Dimension] specify the dimension of the data (e.g; the dimension of a scalar is 0, of an array 1, etc).
  \item[Unit] provide the unit of the data and the data-system of units (cgs or mks).
  \end{description}
}
\fi

\begin{table}[htbp]
  \centering
  \caption{Output parameters}
  \label{tab:output}
  \begin{tabular}{lcccc}
    \toprule
    Name & Status & Data type & Dimension & Unit \\
    \midrule
    %x & mandatory/optional & x & x & x \\
    DP5\_SAS\_MASS & mandatory & float & 0 & M$_\odot$ \\
    DP5\_SAS\_MASS\_STD & mandatory & float & 0 & M$_\odot$ \\
    DP5\_SAS\_MASS\_METADATA & mandatory & string & 1 & N/A \\
    DP5\_SAS\_RADIUS & mandatory & float & 0 & R$_\odot$ \\
    DP5\_SAS\_RADIUS\_STD & mandatory & float & 0 & R$_\odot$ \\
    DP5\_SAS\_RADIUS\_METADATA & mandatory & string & 1 & N/A \\
    DP5\_SAS\_AGE & mandatory & float & 0 & Gyr \\
    DP5\_SAS\_AGE\_STD & mandatory & float & 0 & Gyr \\
    DP5\_SAS\_AGE\_METADATA & mandatory & string & 1 & N/A \\
    \bottomrule
  \end{tabular}
\end{table}


% ~~~~~~~~~~~~~~~~~~~~~~~~~~~~~~~~~~~~~
% Processing description
% ~~~~~~~~~~~~~~~~~~~~~~~~~~~~~~~~~~~~~
\clearpage
\chapter{Processing description}
\label{chap:processing}

\section{Type of delivery}
\label{sec:delivery}

Prototype

\iffalse
% TEXT FROM TEMPLATE BELOW IN emph OR itshape
{
  \itshape

  The algorithms which are specified for the pipeline can come in different
  shapes and forms. We expect that the specifications will be delivered to the
  PDC under different forms, accordingly. Please indicate which one applies to
  the specified algorithm:

  \begin{itemize}
    \firmlist
  \item a legacy code. In that case, please contact WP12 office because some
    quality requirements are needed.
  \item a prototype. In that case, and in the first version of this document, no
    pseudo-code is to be provided.
  \item a pseudo-code. If no prototype exists, a workflow describing the
    algorithm main steps and a detailed pseudo-code is needed for
    implementation.
  \end{itemize}
}
\fi


\section{Algorithm maturity}
\label{sec:mature}

Algorithm concept defined, but interfaces (inputs/outputs) unstable. 
Has been tested with randomly-generated pseudo inputs, but needs to be tested with actual inputs from all of the PLATO modules. 

% TEXT FROM TEMPLATE BELOW IN emph OR itshape
\iffalse
{
  \itshape

  Please specify the maturity level of the algorithm and do not hesitate to
  provide any further information on the current status of the algorithm. The
  convention for algorithm maturity is defined as:

  \begin{itemize}
    \firmlist
  \item algorithm not defined
  \item algorithm concept defined, but interfaces (inputs/outputs) unstable
  \item algorithm concept defined and interfaces (inputs/outputs) stable, but
    not all processing steps stable
  \item no change or only minor changes expected
  \end{itemize}

}
\fi

\section{Algorithm source}
\label{sec:source}

The implemented algorithm and test cases are shipped directly to WP12 office alongside this document as a compressed archive.

% TEXT FROM TEMPLATE BELOW IN emph OR itshape
%\emph{If you provide either a prototype or legacy code: link to the source (you
%  can also send the source files directly to the WP12 office) and any useful
%  information for running the code}


\section{Pseudo-code}
\label{sec:pseudo}

N/A

% TEXT FROM TEMPLATE BELOW IN emph OR itshape
%\emph{If no prototype or legacy code is provided, a pseudo-code is needed.
%  The level of detail should should be high enough to allow for the
%  implementation by the PDC}


\section{Flow diagram}
\label{sec:flowchart}

N/A

% TEXT FROM TEMPLATE BELOW IN emph OR itshape
%\emph{Depending of the complexity of the algorithm, a flow diagram could be
%  useful for implementation. If the algorithm comprises several steps and
%  iterative processes, please provide such a diagram. It is expected to show
%  inputs, outputs and processing steps, including decisions and parallel
%  processing branches, if applicable.}


% ~~~~~~~~~~~~~~~~~~~~~~~~~~~~~~~~~~~~~
% Testing
% ~~~~~~~~~~~~~~~~~~~~~~~~~~~~~~~~~~~~~
\clearpage
\chapter{Test case(s)}
\label{chap:tests}


\section{Implementation test case(s)}
\label{sec:test-implement}


% TEXT FROM TEMPLATE BELOW IN emph OR itshape
{
  \itshape

%  Test cases are necessary for the technical validation of the implementation
%  by the PDC . Given input(s) you provide, the PDC will test if it obtains
%  exactly the same output(s). One or several technical reference test(s) shall
%  be provided to allow the PDC for checking the implementation for the different
%  configurations of the algorithm. For each test case, the following must be
%  provided:
}


The test cases are the same as MSAP5\_31 and MSAP5\_32.  These tests are run in \texttt{MSAP5-34-validation.ipynb}. 

Case 1
  \begin{itemize}
      \firmlist
      \item All consistent measurements. 
      \item Inputs: Defaults
      \item Outputs: [(1.0, 0.04, 'IDP\_SAS\_MASS\_GRID\_MIXED'), (0.999, 0.011, 'IDP\_SAS\_RADIUS\_GRID\_MIXED'), (4.6, 0.5, 'IDP\_SAS\_AGE\_GRID\_MIXED')]
  \end{itemize}

Case 2
  \begin{itemize}
      \firmlist
      \item One inconsistent mass measurement. Additionally, in this case, the seismic radius measurement is missing.
      \item Inputs: Defaults, except 1 is added to all the samples from the first mass method in order to artificially create an inconsistent measurement 
      \item Outputs: [None, (1.0005, 0.0092, 'IDP\_SAS\_RADIUS\_GRID\_SURF\_IND'), (4.6, 0.5, 'IDP\_SAS\_AGE\_GRID\_MIXED')]
  \end{itemize}

Case 3
  \begin{itemize}
      \firmlist
      \item Two inconsistent radius measurements. Additionally, in this case, the seismic and granulation mass measurements are missing. 
      \item Inputs: Defaults, except $0.5$ is added to all the samples from the first radius method, and 1 is added to all the samples from the second radius method
      \item Outputs: [(1.002, 0.042, 'IDP\_SAS\_MASS\_FREQS'), None, (4.6, 0.5, 'IDP\_SAS\_AGE\_GRID\_MIXED')]
  \end{itemize}

Case 4
  \begin{itemize}
      \firmlist
      \item Three inconsistent age measurements. 
      \item Inputs: Defaults, except $2,4,6$ are added to the first, second, third age methods 
      \item Outputs: [(1.0, 0.04, 'IDP\_SAS\_MASS\_GRID\_MIXED'), (0.999, 0.011, 'IDP\_SAS\_RADIUS\_GRID\_MIXED'), None]
  \end{itemize}

Case 5
\begin{itemize}
    \firmlist
    \item Consistent but invalid measurements. 
     \item Inputs: Defaults, except the radii are 10 solar radii larger 
     \item Outputs: [None, None, None]
\end{itemize}

\section{Scientific test case(s)}
\label{sec:test-science}

Simulated data would be highly valuable in testing the algorithm. 

% TEXT FROM TEMPLATE BELOW IN emph OR itshape
%\emph{Please specify here and describe if you need or will need simulated data
%  to be produced or provided by the WP12 office for the scientific validation of
%  the performances of thee algorithm.}

\bibliography{references}

\end{document}
